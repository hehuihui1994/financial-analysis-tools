\documentclass[12pt,a4paper]{article}
\usepackage{CJK, CJKpunct}
\usepackage{graphicx}
\usepackage{multicol}
\usepackage{subfigure}
\usepackage{amsmath}
\usepackage{listings}
\lstset{language=C++}
\lstset{breaklines}
\lstset{extendedchars=false}
\usepackage{indentfirst}
\usepackage{amsfonts,amssymb}
\usepackage[numbers]{natbib}
\usepackage{tikz}

\newcommand{\song}{\CJKfamily{song}}
\newcommand{\hei}{\CJKfamily{hei}}
\newcommand{\kai}{\CJKfamily{kai}}
\newcommand{\fs}{\CJKfamily{fs}}

\begin{document}
\begin{CJK*}{UTF8}{song}

\title{Capacity-Constrained Multi-Agent Markov Random Walk on Graphs for Clustering}
\author{朱恬骅 \\
\small{09300240004@fudan.edu.cn}}
\date{\today}
\maketitle

\section{Markov Random Walk}
Suppose that we have a graph $G=(V, E)$, where $|V| = n$. In a Markov Random Walk, typically, the 1-order Markov RW, we have a matrix $M = (m_ij)_{n \times n}$, where each element $m_ij$ denotes the probability of jumping from source vertex $i$ to the destination $j$, and $m_ii$ tells about the probability of staying still at vertex $i$.

The serial $\{M^i\}$ converges to $M_0$ where the connectivity of 


\section{Multi-agent Random Walk}
In \cite{Alamgir:2010:MRW:1933307.1934543}, the authors proposed a model called Multi-agent Random Walk. Informally, they introduced $a$ agents constrained by ropes lengthened $l$. In the space of states $S = V^a$, each state $s = (v_1, v_2, ..., v_a) \in S$ denotes the positions of agents. And the transition matrix for jumping from $s = {s_1, ..., s_a}$ to $t = {t_1, ..., t_a} \in S$ is defined as:

\begin{equation}
M(s, t) := \left\{
    \begin{array}{ll}
        0, & \text{if } \exists i \text{ such that dist}(s_i, t_i) > l \\
        \sum_{\sigma \in S_a}{\prod\limits_{i=1}^{a}{P(s_\sigma(i), t_j)}}, & \text{Otherwise} \\
    \end{array}
    \right.
\end{equation}

In the equation above, $\sigma$ runs over the set $S_a$ of all permutations of $a$ elements. $\text{dist}(s, t)$ is a distance function, which defines the distance between two vertices with regard to $l$. This definition is fine but impractical. To make MARW a practical algorithm, the authors concedes to try simple random walk at first and then check it for the rope criterion to decide whether to have another try or to accept the result. This try-fail method sounds suitable for graphs where its nodes have lower degrees. For those with higher degrees, the authors suggest to move agents one-by-one in a randomized order.

\section{Capacity constraints}
In network flows, we have defined a function called capacity function, that is, $C: V \times V \rightarrow \mathbb{R}$.

We here introduce the capacity constraints into our multi-agent random walking process. Here we define $C$ as a capacity function having its value domain restrained on $\mathbb{N}$. $C(s,t)$ defines the maximal agents allowed to travel from vertex $s$ to $t$ in one iteration. 

Hence, we have three values (and constraints) on our graph. The first is the transition probability function $P: V \times V \rightarrow [0, 1]$. The second is the `distance' between vertices. And the last one is our capacity constraints.

Physically, the three functions (or constraints) have different meaning, each of which characterizes a feature of vertices. 

\section{方法}


\section{实验结果}


\subsection{生成关键词词典}


\subsection{词语的聚类}


\section{小结}


\renewcommand\refname{References}
\bibliography{main}
\bibliographystyle{plain}

\end{CJK*}
\end{document}