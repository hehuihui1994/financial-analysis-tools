\documentclass[12pt]{article}
\usepackage{CJK}
\usepackage{graphicx}
\usepackage{multicol}
\usepackage{subfigure}
\usepackage{amsfonts,amssymb}
\usepackage{listings}

\begin{CJK*}{UTF8}{song}
\CJKindent

\begin{document}
\title{股票数据分析汇总报告}
\author{朱恬骅}
\date{\today}
%\institute{09300240004}
%\frame{\titlepage}

\maketitle

\section{数据集}
\subsection{小规模数据集}
在银行、医药、钢铁板块各选择四支股票,构成12支股票数据集(S12)。
\begin{multicols}{2}
\begin{enumerate}
\item 银行:
\subitem 600000 浦发银行
\subitem 600015 华夏银行
\subitem 600016 民生银行
\subitem 600036 招商银行
\item 医药:
\subitem 600085 同仁堂
\subitem 600129 太极集团
\subitem 6000422 昆明制药
\subitem 000423 东阿阿胶
\item 钢铁:
\subitem 600019 宝钢股份
\subitem 000959 首钢股份
\subitem 000709 河北钢铁
\subitem 600569 安阳钢铁
\end{enumerate}
\end{multicols}

\subsection{中等规模数据集}
\subsubsection{来自5个不交叉板块的50支股票}
选择了来自银行、钢铁、医药、酒、软件这五个板块的50支股票,构成不交叉板块50支股票数据集(S50)。
\begin{multicols}{2}
\begin{enumerate}
\item 银行:
\subitem 000001 深发展A
\subitem 002142 宁波银行
\subitem 600000 浦发银行
\subitem 600015 华夏银行
\subitem 600016 民生银行
\subitem 600036 招商银行
\subitem 601009 南京银行
\subitem 601166 兴业银行
\subitem 601169 北京银行
\subitem 601288 农业银行
\item 钢铁:
\subitem 000629 攀钢钒钛
\subitem 000709 河北钢铁
\subitem 000717 韶钢松山
\subitem 000898 鞍钢股份
\subitem 000932 华菱钢铁
\subitem 600282 南钢股份
\subitem 600019 宝钢股份
\subitem 000959 首钢股份
\subitem 000022 济南钢铁
\subitem 600569 安阳钢铁
\item 医药:
\subitem 601607 上海医药
\subitem 600511 国药股份
\subitem 600833 第一医药
\subitem 600713 南京医药
\subitem 000028 国药一致
\subitem 600085 同仁堂
\subitem 600129 太极集团
\subitem 600422 昆明制药
\subitem 000423 东阿阿胶
\subitem 002589 瑞康医药
\item 酒:
\subitem 000568 泸州老窖
\subitem 000858 五粮液
\subitem 600519 贵州茅台
\subitem 600779 水井坊
\subitem 000596 古井贡酒
\subitem 600809 山西汾酒
\subitem 000799 酒鬼酒
\subitem 002304 洋河股份
\subitem 600702 沱牌舍得
\subitem 600559 老白干酒
\item 软件:
\subitem 600570 恒生电子
\subitem 600756 浪潮软件
\subitem 000948 南天信息
\subitem 600271 航天信息
\subitem 002063 远光软件
\subitem 002065 东华软件
\subitem 000938 紫光股份
\subitem 002073 软控股份
\subitem 002090 金智科技
\subitem 002230 科大讯飞
\end{enumerate}
\end{multicols}

\subsubsection{来自5个交叉板块的50支股票}
选择了来自钢铁、煤炭、汽车、航空、电力这5个板块的各10支股票。它们都存在一定的相关性,如钢铁和电力都依赖煤炭,汽车依赖钢铁,航空则与煤炭、电力所代表的能源产业有密切关联。构成交叉板块50支股票数据集(R50)。

\begin{multicols}{2}
\begin{enumerate}
\item 钢铁:
\subitem 000629 攀钢钒钛
\subitem 000709 河北钢铁
\subitem 000717 韶钢松山
\subitem 000898 鞍钢股份
\subitem 000932 华菱钢铁
\subitem 600282 南钢股份
\subitem 600019 宝钢股份
\subitem 000959 首钢股份
\subitem 000022 济南钢铁
\subitem 600569 安阳钢铁
\item 煤炭:
\subitem 000780 平庄能源
\subitem 000723 美锦能源
\subitem 002128 露天煤业
\subitem 600188 兖州煤业
\subitem 600348 阳泉煤业
\subitem 600546 山煤国际
\subitem 600740 山西焦化
\subitem 601001 大同煤业
\subitem 601666 平煤股份
\subitem 601898 中煤能源
\item 汽车:
\subitem 000550 江铃汽车
\subitem 000572 海马汽车
\subitem 000625 长安汽车
\subitem 000800 一汽轿车
\subitem 000868 安凯客车
\subitem 000927 一汽夏利
\subitem 000951 中国重汽
\subitem 000957 中通客车
\subitem 002594 比亚迪
\subitem 600006 东风汽车
\item 航空:
\subitem 600029 南方航空
\subitem 600115 东方航空
\subitem 600221 海南航空
\subitem 600316 洪都航空
\subitem 000089 深圳机场
\subitem 600004 白云机场
\subitem 600009 上海机场
\subitem 600151 航天机电
\subitem 600893 航空动力
\subitem 000901 航天科技
\item 电力:
\subitem 600011 华能国际
\subitem 600021 上海电力
\subitem 600027 华电国际
\subitem 600644 乐山电力
\subitem 600101 明星电力
\subitem 600116 三峡水利
\subitem 600131 岷江水电
\subitem 600236 桂冠电力
\subitem 600292 九龙电力
\subitem 600310 桂东电力
\end{enumerate}
\end{multicols}

\subsection{大规模数据集}
\subsubsection{100支股票数据集}
选择历史数据最多的100支股票构成100支股票数据集(S100);在所有股票中随机选择500支股票构成500支股票数据集(S500)。具体的股票名单从略。

\subsection{数据集中股价特征的时间段}
对于小规模的数据集,分别选择了2007年7月1日~2009年7月1日、2008年7月1日~2010年7月1日、2009年7月1日~2011年7月1日这三个时间段的数据,分别记为07、08、09,如12支股票数据集在2007~2009年间的数据,记为S12-07。对于两个中等规模的数据集,选择了2009年7月1日~2011年7月1日和2010年7月1日~2012年7月1日这段时间的数据。对于两个大规模数据集,选择2010年1月1日~2012年1月1日这两年的数据。这样我们实际上共有11个不同时间、不同板块的数据集。

\section{单视图下的股票聚类实验}
\subsection{股价信息}
\subsubsection{表示法}
\paragraph{时序涨跌幅词频表示法}
将每一支股票的走势看作一篇文档。设每支股票取$T+1$天的价格信息,建立一个大小为$2T$的词汇表,包括了“第$i$天涨1\%”和“第$i$天跌1\%”,$i=2,3,...,T+1$。将股票的走势表现为这$2T$个词上的词频。

\paragraph{股票涨跌幅词频表示法}
将每一天的走势看作一篇文档,设共有$N$支股票,则建立一个大小为$2N$的词汇表,包括“第$i$支股票涨1\%”和“第$i$支股票跌1\%”,$i=1,2,...,N$。将每天的股票行情表示为这$2N$个词上的词频。

\paragraph{正负零收益表示法}
将每一支股票的走势看作一篇文档。设每支股票取$T+1$天的价格信息,建立一个大小为$3T$的词汇表,包括了“第$i$天涨”、“第$i$天持平”和“第$i$天跌”,$i=2,3,...,T+1$。将股票的走势表现为这$3T$个词上的词频。

\subsubsection{聚类方法}
\label{SSSAggregationMethods}
\paragraph{直接K-Means法}
对于选定的股价特征,直接运行K-Means。由于初始中心的随机性,运行多次,选取类与类之间分布最为平均的一次结果。

\paragraph{LDA法}
对选定的股价特征,运行LDA进行聚类,选取最可能属于的topic作为这支股票的类标记。

\paragraph{LDA + K-Means法}
对选定的股价特征,先运行LDA进行聚类;将该股票属于这些topic的可能性作为新的特征,运行K-Means进行聚类。

\subsection{文本信息(经营范围描述)}
\subsubsection{表示法}
\paragraph{全文词频表示法}
即对经营范围描述信息进行分词后,对所有出现的词都计算词频,是最简单的方法。

\paragraph{构建关键词词典}
为去除一些意义不大的高频词,需要构造一个比较干净的关键词词典。第一种方法是计算一个词的文档间频率DF及其对应的信息熵$H(w)$,进行降序排序,这就构建出了针对特定语料的关键词词典。以这一词典为基础统计的全文词频,将比在所有词或高频词的字典上统计得到的词频更能代表语料的特征。

\subsubsection{聚类方法}
同 \ref{SSSAggregationMethods} 。

\subsection{结果}
在所有2215支股票的经营范围描述文本信息中,采用构建关键词词典方法,查找出的前100个高频词如下:

\begin{multicols}{5}
经营\\
技术\\
销售\\
生产\\
业务\\
出口\\
设备\\
服务\\
开发\\
进出口\\
材料\\
许可\\
企业\\
项目\\
机械\\
工程\\
国家\\
商品\\
加工\\
咨询\\
除外\\
制造\\
本企业\\
投资\\
配件\\
电子\\
禁止\\
管理\\
公司\\
不含\\
进出口业\\
进出口业务\\
相关\\
许可证\\
化工\\
设计\\
代理\\
机械设备\\
系统\\
仪器\\
建筑\\
营本企业\\
计算机\\
经营本企业\\
金属\\
规定\\
信息\\
汽车\\
法律\\
仪表\\
法规\\
范围\\
货物\\
安装\\
公司经营\\
原辅\\
贸易\\
零配件\\
辅材料\\
所需的
\end{multicols}

在所有可能的词(组)中,信息熵最大的词(组)如下:

\begin{multicols}{5}
咨询\\
国家\\
材料\\
机械\\
除外\\
商品\\
禁止\\
项目\\
进出口\\
进出口业\\
服务\\
制造\\
业务\\
企业\\
开发\\
设备\\
公司\\
投资\\
管理\\
加工\\
不含\\
许可\\
许可证\\
工程\\
相关\\
电子\\
自产\\
设计\\
仪器\\
化工\\
产品\\
生产\\
代理\\
仪表\\
限定\\
技术\\
建筑\\
零配件\\
制品\\
范围\\
规定\\
进口\\
货物\\
租赁\\
安装\\
信息\\
房地产\\
计算机\\
及其\\
各类\\
法律\\
批发\\
法规\\
配件\\
自营\\
零售\\
汽车\\
限制\\
系统
\end{multicols}

然后使用贝叶斯平均方法提取了所有2215支股票经营范围描述的关键词。得到的部分结果如下:

000001  监管 人民币 汇款 借款 放款 非贸易 有价证券 汇兑 信托业 外币 见证 资信 承兑 各项 存款 贴现 调查 票据 结算
外汇 代理业 人民 保险 境内 买卖 允许 发行 有关 境外 办理\\
002142  十一 十三 十二 金融债 中国银 公众 银行卡 信用证 发放 中期 款项 长期 收付 兑付 短期 吸收 债券 监督 保险业
政府 承兑 银行 拆借 存款 担保 中国 贴现 保管 同业 票据\\
600000  外汇 托管 保险箱 全国 离岸 保障 外币 借款 汇款 兑换 委员会 社会 银行业 见证 资信 中国银 拆借 结汇 股票 存款
担保 公众 贴现 同业 信用证 发放 中期 款项 长期 收付\\
600015  金融债 委员会 中国银 结汇 公众 银行卡 债券 信用证 发放 款项 中期 长期 收付 兑付 短期 政府 吸收 监督 承兑
拆借 存款 担保 贴现 保管 同业 买卖 票据 结算 贷款 代理业\\
600016  本行 十四 十一 十三 十二 可以 银行业 结汇 金融债 公众 银行卡 信用证 发放 中期 款项 长期 收付 兑付 短期 吸收
监督 保险业 承兑 银行 拆借 债券 存款 担保 中国 政府\\
这几个都是银行股。

000028  医用 区域性 救护车 口腔科 化验 缝合 一次性 灭菌 诊断 同化 第一 器具 激素 手术室 急救室 精神 抗生素 射线
麻醉药 超声 敷料 附属 临床 诊疗 蛋白 毒性 疫苗 分析 消毒 合剂\\
000423  膏剂 合剂 糖浆 口服液 保健 颗粒剂 胶囊 药品 批准 食品 范围 许可证 进出口业 商品 生产 销售\\
002589  保存 常温 毒液 罂粟 助听器 隐形眼镜 同化 激素 体外 麻醉药 蛋白 毒性 疫苗 健身器 护理 诊断 抗生素 精神 配送
三类 化学药 饮片 日用品 生化 试剂 中药材 制毒 生物制品 中成药 化妆品\\
600085  营养液 老年病 乌鸡 作用 妇产科 儿科 梅花鹿 乌骨鸡 外科 冷食品 中医科 内科 马鹿 涂膜剂 同仁 皮肤科 供暖 定型
皮肤 北京 诊疗 其中 股份 动植物 西药 饲养 有限公司 图书 保健 饮片\\
600129  执业 中草药 旅馆 水产 西药 作业 二级 首饰 前不 副食品 保健 金银 土地 中成药 养殖 以下 工艺美术 维护 种植
经济 印刷 不得 百货 医疗 旅游 器械 出租 自有 化学 包装\\
这几个都是医药方面的。

但也有不好的例子,比如东华软件的:\\
002065  决定 国务院 机关 注册 选择 行政 自主 工商 登记 不得 活动 开展 批准 后方 法规 法律 审批 自营 规定 各类 限定
许可 代理 禁止 进出口业 公司 管理 商品 除外 进出口\\
完全没有体现出它经营范围是什么。

\section{双视图下的股票聚类实验}
\subsection{视图间聚类的相关性}
对于同一对象的描述,因其所选择的视图有所不同,会导致观察到的结果反映了同一对象的不同特征。然而,由于这些特征反映的是同一对象,我们有理由认为这些特征之间可能存在一定的相关性,而不是如之前的模型所假设的那样是独立的。因此,利用这些特征之间的相关性,以组合不同视图下观察到的数据,并以此进行聚类,可能会得到比简单地将两组特征合并进行聚类更好的结果。

矩阵拆分是建立在词频模型上的。将观察到的数据用词频的形式表示。

\subsection{不同视图下聚类结果的合并}


\subsection{结果}

\end{CJK*}
\end{document}